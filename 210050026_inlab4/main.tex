\documentclass{beamer}

\usepackage[utf8]{inputenc}
\usepackage{hyperref}
\usepackage{amsmath}
\hypersetup{colorlinks=true,linkcolor=blue,urlcolor=cyan}

\usetheme{Madrid}
\usecolortheme{default}

\title[\LaTeX \hspace{0.2ex} Basics \& Advanced] 
{CS 251 - Lab 4 \\ \LaTeX \hspace{0.2ex} Basics \& Advanced}


\author[ATISHAY JAIN] 
{Atishay Jain}

\titlegraphic{
    \includegraphics[width = 1.8 cm]{iitb.png}
}

\institute[IITB] 
{
  IIT Bombay
}

\date[\hspace{2 cm}2022]
{August 2022}

\begin{document}

\frame{\titlepage}

\begin{frame}{Introduction of myself}
    Something about myself, I am Atishay Jain, a 19-year old, humble sophomore studying Computer Science \& engineering at IIT Bombay. Currently, I am living in Hostel 16, IIT Bombay, but my hometown is Tikamgarh, Madhya Pradesh. My interests include playing Keyboard, Cricket, Badminton, Carrom, Coding \& Programming. I'm always keen to explore more about Programming and learn new things. I also like to go to gym now a days. Right now, I am learning many different important Softwares that programmers use quite often, as part of SSL lab. By the way, I have my own website, which was made a few days ago, \textit{also as a part of SSL lab}. To know more about me, you can check out my \href{https://www.cse.iitb.ac.in/~atishay/}{website}. 
\end{frame}

\begin{frame}{Table of Contents}
    \tableofcontents
    Note how the links are redirecting to the corresponding page
\end{frame}

\section{Introduction}
\begin{frame}{Introduction}
    We first see the power of frames in \textbf{\LaTeX}. We dont need to write each and every slide just for a new line.
    \pause
    We can just use beamer class with the feature of pauses.
    \pause
    However, \textbf{\LaTeX} has another ( rather the most important usage ), namely the use \textcolor{red}{formatting text} in a more mathematical way.
\end{frame}

\section{Equations}
\begin{frame}{Equations}
    We can write many equations, can be labelled like the following
    \begin{equation}
        e^{i\alpha} = cos(\alpha) + i \; sin(\alpha)
    \end{equation}
    \pause
    or the unlabelled equations like the force between two charges given by
    \begin{equation*}
        F = \frac{1}{4\pi\epsilon_0}\frac{q_1q_2}{r^2}
    \end{equation*}
\end{frame}

\section{Itemize and Linking}
\begin{frame}{Itemize and Linking}
    Also, \LaTeX can be used to present the items in a list format, for example, some common ways of sorting an array are:
    \begin{itemize}
        \item Bubble sort
        \item Insertion sort \pause , then there are the more rigorous algorithms like
        \item QuickSort
        \item Heap sort \pause , \textit{and then the best known algorithm}
        \item \textcolor{red}{Monkey sort} (or) Bogo-sort
    \end{itemize}
    Some pointers to the last algorithm can be found at \href{https://en.wikipedia.org/wiki/Bogosort}{here}
\end{frame}

\section{Matrices}
\begin{frame}{Matrices}
    We can also write matrices in \LaTeX, for example the identity matrix of size (3x3) is
    \begin{equation*}
        I_3 = 
        \begin{bmatrix} 
            1 & 0 & 0 \\
            0 & 1 & 0 \\
            0 & 0 & 1 \\
        \end{bmatrix}
    \end{equation*}
    \pause
    \textcolor{red}{Bonus: try to indent like the below equation}
    \begin{equation*}
        \begin{split}
            (\mathbf{a}\cdot\mathbf{b})^2 &= (\sum a_ib_i)^2 \\ &\leq (\sum a_i^2)(\sum b_i^2) 
        \end{split}
    \end{equation*}
\end{frame}

\end{document}